\documentclass[12pt]{article}

\usepackage[margin=1.1in,footskip=.25in]{geometry}

\usepackage{amsmath, amssymb}
\usepackage{mathtools}

\usepackage[most]{tcolorbox}

\tcbset{
    %frame code={}
    %center title,
    colback=gray!5!white,
    colframe=gray!75!black,
    toptitle=2mm,
    righttitle=2mm,
    bottomtitle=2mm,
    fonttitle= \bfseries\large,
    left=10pt,
    right=10pt,
    top=10pt,
    bottom=10pt,
    %colback=gray!5,
    %colframe=gray,
    width=\dimexpr\textwidth\relax,
    %enlarge left by=0mm,
    boxsep=5pt,
    %arc=0pt,outer arc=0pt,
}

\usepackage{xepersian}
\settextfont[Scale=1]{Vazir}

\setlength{\parindent}{0pt}

\renewcommand{\baselinestretch}{1.5} 

\begin{document}


\begin{tcolorbox}[
title=تبصره 1
]
هرگونه خط خوردگی ، یا نوشتار دستی در قرارداد باعث ابطال آن می شود 
\end{tcolorbox}




\section{مشخصات طرفین}


\subsection{کارفرما}


شرکت  . . . . . . . . . . .  شماره ثبت شرکت . . . . . . . . . . . . .

\subsection{کارمند}


آقا . . . . . . . . . . . . . . فرزند . . . . . . . . . .  کد ملی . . . . . . . . . . .



\section{نوع کار}


\begin{itemize}
	\item موضوع قرارداد طراحی و تولید وب سایت و اپلیکیشن های موبایل می باشد
	\item  پس از اینکه طراحی و برنامه نویسی نرم افزار انجام شد ، 
	نرم افزار متعلق به کارفرما می باشد اما برنامه نویس می تواند از برنامه به عنوان رزومه کاری خود یاد کند
	\item در صورت ناتوانی برنامه نویس در انجام تعهدات نسبت به نرم افزار ، حق فسخ قرارداد برای کارفرما محفوظ می باشد
	\item به دلیل گسترده بودن مباحث تولید نرم افزار ، 
	کارفرما نمی تواند به دلیل ناتوانی در تولید نرم افزار ، برنامه نویس را مواخذه یا جریمه کند
	\item برنامه نویس متعهد می شود که پس از اتمام طراحی هر برنامه   ، منابع برنامه را در اختیار کارفرما قرار دهد
\end{itemize}



\section{محل انجام کار}


از آنجایی که فرآیند تولید و طراحی نرم افزار محدود به مکان خاصی نمی باشد ، برنامه نویس می تواند ساعات کاری خود را در مکان های دلخواه خودش مشغول به کار باشد .


\section{تاریخ انعقاد قرارداد}

این قرارداد در تاریخ . . . . . . . . . . . . . منعقد شد

\section{نوع قرارداد}

\begin{tcolorbox}[
title=تبصره 2
]
طبق ماده 7 قانون کار جمهوری اسلامی ایران ، قرارداد کار می تواند به صورت کتبی یا شفاهی باشد بنابراین کارفرما و  کارمند لزومی به امضا زدن ندارند و صرفاً اعتماد به یکدیگر کافی است ، در صورت عدم اعتماد می توانند توافق نکنند
\end{tcolorbox}

نوع قرارداد به صورت موقت می باشد


\section{مدت قرارداد}

\begin{tcolorbox}[
title=تبصره 3
]
طبق ماده 11 قانون کار جمهوری اسلامی ایران ، طرفین قرارداد می توانند مدتی را تحت عنوان دوره آزمایشی تعیین نمایند ، این دوره برای نیروهای متخصص سه ماه می باشد ، در خلال این دوره هر یک از طرفین قرارداد می تواند بدون اخطار قبلی رابطه ی کار را قطع نماید 
\end{tcolorbox}

مدت قرارداد به صورت 6 ماهه می باشد


\section{ساعات کار}

میزان ساعات کار و زمان شروع و خاتمه ی آن ، صرفاً با رضایت طرفین تعیین می گردد


\section{حق السعی}

حقوق ماهیانه کارمند در قبال انجام کار . . . . . . . . . . . . . . . . . . . می باشد .

حقوق و مزایا به صورت ماهیانه به شماره حساب  . . . . . . . . . . . . . .

شعبه . . . . . . . . . . توسط کارفرما یا نماینده قانونی ایشان پرداخت می گردد .


\section{بیمه}

طبق ماده 148 قانون کار جمهوری اسلامی ایران ، کارفرما باید کارمند را بر اساس قانون تامین اجتماعی بیمه کند 




\section{شرایط خاتمه قرارداد}

\begin{tcolorbox}[
title=تبصره 4
]
در هیچ یک از تبصره و ماده های قانون کار جمهوری اسلامی ایران ، حق گرفتن چک و سفته برای کارفرما لحاظ نشده است
\end{tcolorbox}


طبق ماده ی 25 قانون کار جمهوری اسلامی ایران ، چنانچه قرارداد به صورت موقت منعقد شده باشد هیچ یک از طرفین به تنهایی و بدون اخطار قبلی حق فسخ قرارداد را ندارد .

طبق ماده ی 21 قانون کار جمهوری اسلامی ایران ، کارمند می تواند برای خاتمه ی قرارداد از کار خود استعفا دهد ، در این صورت کارمند موظف است تا یک ماه پس از استعفا به کار خود ادامه دهد



\end{document}